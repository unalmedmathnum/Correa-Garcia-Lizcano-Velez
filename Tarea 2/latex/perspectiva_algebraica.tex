\section{Teoría del método:}
El objetivo de este método es el de hallar la función continua que mas se adecue a un conjunto de datos dados de la forma $(x,y)$, de donde, $x$ es la variable independiente y $y$ es una variable que depende de $x$. El método determina la función mas adecuada por medio del criterio de mínimo error cuadrático:\\

Sea $A:=\{(x_k,y_k)\}_{k=1}^n$ el conjunto de par ordenados de los datos dados, donde $y_k$ depende de $x_k$. Y sea $\{f_j(x)\}_{j=0}^m$ un conjunto de funciones linealmente independiente.\\

Queremos hallar una función $f(x)$ continua tal que:
$$f(x)=\sum_{j=0}^mc_jf_j(x)$$
$$f(x_k)\approx y_k$$
De donde, $c_j\in\mathbb{R}$ y el criterio de aproximación esta dado por el mínimo error cuadrático, como se muestra a continuación:.\\

\subsubsection{Mínimo error cuadrático:}
Definimos el residuo para cada punto como $$e_k:=y_k-f(x_k)$$
El criterio de mínimo error cuadrático nos dice que minimicemos la expresión dada por: $$E_{cm}(f)=\sqrt{\frac{1}{n}\sum_{k=1}^ne_k^2}$$
Lo cual es equivalente a minimizar la expresión:
$$E_c(f)=\frac{1}{n}\sum_{k=1}^ne_k^2$$
\section{Perspectiva algebraica:}
Para implementar el método debemos tener en cuenta qué es lo que se quiere hallar:  $c_j\in\mathbb{R}$ $\forall j\in\{1,2,...,m\}$ tal que $E_c(f)$ sea el mínimo. Esto lo podemos hacer através de derivar la expresión de $E_c(f)$ con respecto a cada coeficiente $c_j$.\\

Primero escribamos $E_c(f)$ en términos de las funciones $\{f_j(x)\}_{j=1}^m$ tales que $f=\sum_{j=1}^mc_jf_j$
$$E_C(f)=\frac{1}{n}\sum_{k=1}^ne_k^2=\frac{1}{n}\sum_{k=1}^n\left(y_k-f(x_k)\right)^2=\frac{1}{n}\sum_{k=1}^n\left(y_k-\sum_{j=1}^mc_jf_j(x_k)\right)^2$$
Ahora, derivemos esta expresión con respecto a $c_i$ para cada $i\in\{1,2,...m\}$. Es fácil ver que queda de la siguiente forma: 
$$\frac{\partial E_c}{\partial c_i}=-\frac{2}{n}\sum_{k=1}^n\left(y_k-\sum_{j=0}^mc_jf_j(x_k)\right)(f_i(x_k))$$
Luego, al igual a cero $\frac{\partial E_c}{\partial c_i}=0$ para $i\in\{1,2,...m\}$, quedamos con $m$ ecuaciones y $m$ incógnitas de la forma:
$$\sum_{k=1}^ny_kf_i(x_k)=\sum_{k=1}^n\left(\sum_{j=1}^mc_jf_j(x_k)\right)f_i(x_k)=\sum_{j=1}^m\left(\sum_{k=1}^nf_j(x_k)f_i(x_k)\right)c_j$$
Que es equivalente a resolver la siguiente ecuación de matrices y vectores:
$$
\begin{bmatrix}
    \sum_{k=1}^n(f_1(x_k))^2 & \sum_{k=1}^nf_1(x_k)f_2(x_k) & \dots & \sum_{k=1}^nf_1(x_k)f_m(x_k) \\
    \sum_{k=1}^nf_2(x_k)f_1(x_k) & \sum_{k=1}^n(f_2(x_k))^2 & \dots & \sum_{k=1}^nf_2(x_k)f_m(x_k) \\
    \vdots & \vdots & \ddots & \vdots\\
     \sum_{k=1}^nf_m(x_k)f_1(x_k) & \sum_{k=1}^nf_m(x_k)f_2(x_k) & \dots & \sum_{k=1}^n(f_m(x_k))^2 \\
\end{bmatrix}
\begin{bmatrix}
    c_1\\c_2\\\vdots\\c_m\\
\end{bmatrix}
=
\begin{bmatrix}
    \sum_{k=1}^n f_1(x_k)y_k\\\sum_{k=1}^n f_2(x_k)y_k\\\vdots\\\sum_{k=1}^n f_m(x_k)y_k\\
\end{bmatrix}
$$
Hallando el vector del lado izquierdo de la ecuación, podemos obtener los valores deseados para hallar $f$ -  la mejor aproximación según el criterio de aproximación cuadrático.



